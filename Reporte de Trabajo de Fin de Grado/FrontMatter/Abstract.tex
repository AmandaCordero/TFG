\begin{abstract}
	Las enfermedades causadas por \textit{Streptococcus pneumoniae} representan un importante problema de salud pública, especialmente en lactantes. Este trabajo presenta el desarrollo de un simulador computacional del sistema inmune infantil para predecir la efectividad de la vacuna antineumocócica conjugada Quimi-Vio7\textregistered. El modelo implementado utiliza un enfoque basado en agentes para representar los componentes clave del sistema inmune, incluyendo células B, antígenos y la dinámica de los centros germinales. Se emplearon datos reales de ensayos clínicos con la vacuna VCN7-T para validar el modelo, utilizando la distancia de Chamfer como métrica de similitud entre los resultados simulados y los datos experimentales. 
	% Los resultados muestran que el simulador puede capturar adecuadamente la variabilidad en las respuestas inmunes para diferentes serotipos neumocócicos, con distancias de Chamfer que oscilan entre 2.773 (serotipo 18C) y 26.348 (serotipo 14). 
	Este enfoque computacional ofrece una herramienta prometedora para acelerar el desarrollo y evaluación de vacunas, reduciendo la dependencia de ensayos clínicos extensivos.

\end{abstract}

\begin{enabstract}
    Diseases caused by \textit{Streptococcus pneumoniae} represent a significant public health problem, particularly in infants. This work presents the development of a computational simulator of the infant immune system to predict the effectiveness of the pneumococcal conjugate vaccine Quimi-Vio7\textregistered. The implemented model uses an agent-based approach to represent key immune system components, including B cells, antigens, and germinal center dynamics. Real clinical trial data with the VCN7-T vaccine were used to validate the model, employing the Chamfer distance as a similarity metric between simulated results and experimental data. 
	% The results show that the simulator can adequately capture variability in immune responses for different pneumococcal serotypes, with Chamfer distances ranging from 2.773 (serotype 18C) to 26.348 (serotype 14).
	This computational approach offers a promising tool to accelerate vaccine development and evaluation, reducing dependence on extensive clinical trials.
\end{enabstract}