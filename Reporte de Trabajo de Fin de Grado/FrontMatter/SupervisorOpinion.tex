\begin{opinion}	
\textbf{Título de la tesis:} Simulador de sistema inmune para predecir la respuesta de lactantes a vacunas antineumocócicas conjugadas

\textbf{Autora:} Amanda Cordero Lezcano

El trabajo de diploma presentado por la estudiante Amanda Cordero Lezcano constituye una contribución significativa al campo de la inmunología computacional y al desarrollo de vacunas, al abordar un problema de alto impacto para la salud pública: la predicción de la respuesta inmune en lactantes mediante modelos computacionales.

Este proyecto se enmarca en la colaboración estratégica entre la Facultad de Matemática y Computación de la Universidad de La Habana y el Instituto Finlay de Vacunas (IFV), institución con más de tres décadas de experiencia en el diseño de vacunas innovadoras. Bajo el compromiso social de fortalecer los Programas Ampliados de Inmunización, y aprovechando las lecciones del Proyecto Soberana, esta alianza académico-científica impulsa el uso de herramientas computacionales en el desarrollo de nuevos candidatos vacunales, articulando la excelencia investigativa del IFV con el rigor metodológico de la modelación matemática.

La investigación se desarrolla en un contexto prioritario para Cuba: la necesidad de contar con modelos predictivos eficientes que permitan evaluar formulaciones vacunales como el candidato Quimi-Vio7® antes de emprender ensayos clínicos a gran escala.

Amanda demostró una capacidad excepcional para integrar conocimientos avanzados en inmunología, matemáticas y programación —muchos de ellos fuera del currículo tradicional de la carrera—, destacando por su rigor científico, creatividad metodológica y autonomía intelectual. Diseñó un simulador que modela procesos clave del sistema inmune, como la activación de células B, la dinámica de los centros germinales y la generación de memoria inmunológica.

Entre sus principales aportes técnicos, sobresale la implementación de la selección clonal mediante distribución de Boltzmann, así como el uso de la métrica de Chamfer para la validación cuantitativa del modelo. 
% Estos enfoques, poco convencionales en este dominio, muestran un manejo competente de metaheurísticas y programación concurrente. Para enfrentar los retos del modelado paralelo, resolvió con eficacia problemas de sincronización mediante el uso de mutex.

La herramienta desarrollada en Python integra bibliotecas científicas como NumPy y SciPy, y fue aplicada sobre datos clínicos reales del ensayo PCV7-TT del IFV. El modelo fue validado con resultados experimentales, logrando ajustes prometedores para serotipos como 18C (Chamfer = 2.773) y 5 (3.241). La autora identificó limitaciones (como la menor precisión para el serotipo 14) y propuso mejoras como la inclusión de mecanismos de inmunidad innata, lo cual evidencia una visión crítica y prospectiva.

Durante todo el proceso, Amanda mantuvo una comunicación fluida y proactiva con el equipo de tutoría. Mostró una notable perseverancia frente a las limitaciones de datos y recursos, así como una actitud investigativa sostenida que le permitió apropiarse de contenidos complejos como la inmunología en lactantes, a través del autoaprendizaje y el análisis profundo de la bibliografía especializada.

El trabajo representa un aporte metodológico original en la intersección entre computación y biomedicina, y sienta las bases para una línea de investigación sostenible en biocomputación aplicada a la salud pública cubana.

Amanda ha demostrado ser una investigadora emergente con gran potencial, capaz de combinar creatividad técnica, rigor analítico y compromiso social. En un tiempo limitado y bajo condiciones retadoras, logró alcanzar con solvencia los objetivos propuestos.

Como toda obra humana, su tesis es perfectible —lo cual es natural en una primera versión—, pero como tutores estamos plenamente satisfechos con su desempeño. Consideramos que su trabajo merece la calificación de Excelente, y confiamos en que continuará realizando aportes valiosos al desarrollo de herramientas computacionales en colaboración con instituciones como el IFV.


\begin{flushleft}
	

	
	\underline{\hspace{6.5cm}}\\
	% \qquad La Habana, 4 de enero de 2024.
	
	\qquad MSc. Celia T. González González
	
	\qquad MSc. Wilfredo Morales Lezca
	
	\qquad Facultad de Matemática y Computación
	
	\qquad Universidad de la Habana
\end{flushleft}

\end{opinion}