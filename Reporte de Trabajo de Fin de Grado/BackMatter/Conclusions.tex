%===================================================================================
% Chapter: Conclusiones
%===================================================================================
\chapter*{Conclusiones}\label{chapter:conclusions}
\addcontentsline{toc}{chapter}{Conclusiones}

\begin{itemize}
    \item Se desarrolló un simulador  del sistema inmune de lactantes que modela la respuesta a vacunas antineumocócicas conjugadas, integrando mecanismos inmunológicos clave como la activación de células B, la formación de centros germinales y la generación de memoria inmunológica.

    \item La validación del modelo contra datos clínicos reales demostró su capacidad para predecir patrones de respuesta inmune, aunque con variabilidad entre serotipos. Los mejores ajustes se obtuvieron para los serotipos 18C (distancia Chamfer = 2.773) y 5 (3.241), mientras que el serotipo 14 presentó mayor discrepancia (26.348).

    \item Este trabajo contribuye al campo de la inmunología computacional al proporcionar una herramienta para evaluar candidatos vacunales \textit{in silico}, con potencial para reducir costos y tiempos en el desarrollo de vacunas. 

    \item Como limitación, se identificó que el modelo tiene menor precisión para ciertos serotipos (especialmente el 14), lo que sugiere la necesidad de incorporar mecanismos inmunológicos adicionales en futuras versiones del simulador.
\end{itemize}

