%===================================================================================
% Chapter: Conclusiones
%===================================================================================
\chapter*{Recomendaciones}\label{chapter:recomendations}
\addcontentsline{toc}{chapter}{Recomendaciones}

\begin{itemize}
    \item Dar continuidad al desarrollo del simulador utilizando los resultados una vez realizado el ensayo clínico para el candidato vacunal futuro, QuimiVio-11 (que incluye 4 nuevos serotipos). Esto permitirá fortalecer la confiabilidad predictiva del modelo mediante una validación empírica más amplia.

    \item Implementar técnicas de optimización para abordar posibles casos de subajuste, especialmente en serotipos con mayores distancias de Chamfer (ej. serotipo 14). 
    % Esto podría incluir la ampliación la exploración del espacio de parámetros


    \item Añadir complejidad al modelo mejorando la fidelidad biológica del modelo integrando elementos clave de la inmunidad innata como receptores de reconocimiento de patrones o dinámica de células fagocíticas (neutrófilos, macrófagos).
    

    \item Adaptar el simulador para modelar respuestas inmunes en diferentes poblaciones, especialmente para el adulto mayor, mediante el desarrollo de conjuntos de parámetros específicos por edad y la consideración de inmunosenescencia en población geriátrica.
    

\end{itemize}