%===================================================================================
% Chapter: Marco Teorico
%===================================================================================
\chapter{Preliminares}\label{chapter:preliminares}
%\addcontentsline{toc}{chapter}{Marco Teórico}
%===================================================================================


\textit{Streptococcus pneumoniae} es una bacteria Gram-positiva, encapsulada y con forma lanceolada, que coloniza las vías respiratorias superiores en humanos. Es un patógeno de relevancia clínica global, asociado tanto a enfermedades invasivas (como meningitis, neumonía bacteriémica y sepsis) como a infecciones no invasivas (otitis media aguda y sinusitis).



\section{Características clave}
\begin{itemize}
    \item \textbf{Variabilidad serotípica}: Existen más de 90 serotipos diferenciados por la composición de su cápsula polisacárida, aunque apenas 12 de ellos son responsables de mas del 80\% de las infecciones neumocóccicas invasoras \cite{PREADOJ2001}
    \item \textbf{Poblaciones vulnerables}: Lactantes, adultos mayores (mayores de 65 años) e individuos inmunocomprometidos presentan mayor riesgo de infección grave.
\end{itemize}

\subsection{Importancia de la vacunación}
La diversidad de serotipos y la emergencia de cepas resistentes a antibióticos subrayan la importancia de estrategias preventivas como la vacunación. Las vacunas conjugadas han demostrado reducir significativamente la incidencia de enfermedad neumocócica al inducir inmunidad específica frente a los serotipos más prevalentes \cite{Snedecor2020a,Snedecor2020b}.

\section{Inmunología básica}

El sistema inmunitario humano responde a \textit{Streptococcus pneumoniae} mediante dos grandes componentes:

\begin{itemize}
    \item \textbf{Inmunidad innata}: Primera línea de defensa, compuesta por células como neutrófilos, macrófagos y células dendríticas, así como factores humorales como el sistema del complemento.
    \item \textbf{Inmunidad adaptativa}: Proporciona especificidad y memoria inmunológica, mediada por linfocitos T (CD4$^+$, CD8$^+$) y B.
\end{itemize}

\subsection{Respuesta inmune a \textit{Streptococcus pneumoniae}: linfocitos B, memoria inmunológica y anticuerpos}

Los linfocitos B son fundamentales en la defensa frente a \textit{S. pneumoniae}, especialmente tras la vacunación:
\begin{itemize}
    \item Producen anticuerpos específicos que neutralizan al patógeno.
    \item Experimentan maduración de afinidad y cambio de isotipo (por ejemplo, de IgM a IgG).
    \item Generan células de memoria que permiten una respuesta más rápida y eficaz ante exposiciones futuras.
\end{itemize}

\subsection{Características inmunológicas en lactantes}
\begin{itemize}
    \item \textbf{Anticuerpos maternos}: 
    \begin{itemize}
        \item Transferencia placentaria de IgG (especialmente IgG1) mediante el receptor FcRn.
        \item Protección adicional a través de IgA secretora en la leche materna.
        \item Posible \textit{blunting} (interferencia de anticuerpos maternos con la respuesta vacunal), aunque su impacto clínico es limitado.\cite{AEP2023Blunting}
    \end{itemize}
    \item \textbf{Desarrollo inmune}: El sistema inmunitario neonatal es funcional pero inmaduro, dependiendo inicialmente de la inmunidad materna.
\end{itemize}

\subsection{Definiciones clave}
\begin{itemize}
    \item \textbf{Inmunogenicidad}: Capacidad de una vacuna para inducir una respuesta inmunitaria medible (por ejemplo, producción de anticuerpos).
    \item \textbf{Efectividad clínica}: Protección real conferida contra la enfermedad en condiciones del mundo real.
    \item \textbf{Conjugación proteica}: Las vacunas conjugadas unen polisacáridos capsulares a proteínas transportadoras para potenciar la respuesta inmunitaria, especialmente en niños pequeños.
\end{itemize}

\section{Vacunas antineumocócicas conjugadas}


Las vacunas antineumocócicas conjugadas (PCV) emplean polisacáridos capsulares conjugados a proteínas para inducir una respuesta inmunitaria robusta y duradera y generar memoria inmunológica mediante la activación de linfocitos T y B.


\subsection{Cobertura y correlatos de protección}
Las formulaciones actuales de PCV cubren los serotipos más prevalentes y virulentos. Los correlatos de protección\footnote{Mediciones de parámetros inmunitarios que permiten predecir el grado de protección contra la infección o enfermedad inducida por un patógeno.} pueden variar según el serotipo y la vacuna específica, y son utilizados como criterios en ensayos clínicos y en la evaluación de programas de vacunación \cite{Snedecor2020a}.

\section{Examen ELISA}

El examen ELISA (Ensayo de Inmunoabsorción Ligado a Enzima) es una técnica inmunoenzimática ampliamente utilizada para la detección y cuantificación específica de antígenos o anticuerpos en muestras biológicas. El principio fundamental del ELISA se basa en la unión específica entre un anticuerpo y un antígeno, donde uno de estos componentes está inmovilizado en una superficie sólida, generalmente una placa de microtitulación.

La detección se realiza mediante un anticuerpo conjugado con una enzima, que al reaccionar con un sustrato específico produce un cambio de color proporcional a la cantidad de analito presente en la muestra. Este cambio es medido espectrofotométricamente, permitiendo cuantificar la concentración del antígeno o anticuerpo.

Existen diferentes formatos de ELISA, entre ellos:

\begin{itemize}
    \item \textbf{ELISA directo}: el anticuerpo marcado con enzima se une directamente al antígeno inmovilizado.
    \item \textbf{ELISA indirecto}: utiliza un anticuerpo primario para detectar el antígeno y un anticuerpo secundario marcado con enzima para amplificar la señal.
    \item \textbf{ELISA sándwich}: un anticuerpo captura el antígeno, y otro anticuerpo marcado con enzima detecta el antígeno capturado, aumentando la sensibilidad y especificidad.
    \item \textbf{ELISA competitivo}: el antígeno en la muestra compite con un antígeno marcado por la unión a un anticuerpo específico.
\end{itemize}

Esta técnica es fundamental en diagnóstico clínico, investigación biomédica y control de calidad, debido a su alta sensibilidad, especificidad y rapidez \cite{Lequin2005, Crowther2009}.

