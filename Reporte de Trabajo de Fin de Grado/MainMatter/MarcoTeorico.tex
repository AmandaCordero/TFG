%===================================================================================
% Chapter: Marco Teorico
%===================================================================================
\chapter{Preliminares}\label{chapter:marcoteorico}
%\addcontentsline{toc}{chapter}{Marco Teórico}
%===================================================================================

\section{Contexto microbiológico: \textit{Streptococcus pneumoniae}}

\textit{Streptococcus pneumoniae} (neumococo) es una bacteria Gram-positiva, encapsulada, de forma lanceolada, que coloniza las vías respiratorias superiores en humanos. Es un patógeno de relevancia clínica global, asociado a:

\begin{itemize}
    \item \textbf{Enfermedades invasivas}: Meningitis, neumonía bacteriemica y sepsis.
    \item \textbf{Infecciones no invasivas}: Otitis media aguda y sinusitis.
\end{itemize}

\subsection{Características clave}
\begin{itemize}
    \item \textbf{Variabilidad serotípica}: Posee más de 90 serotipos diferenciados por la composición química de su cápsula polisacárida, siendo aproximadamente 20 los responsables del 80\% de las enfermedades invasivas.
    \item \textbf{Mecanismos de patogenicidad}:
    \begin{itemize}
        \item Evasión inmunológica mediante la cápsula polisacárida (resistencia a la fagocitosis).
        \item Producción de toxinas como la neumolisina.
        \item Adhesinas que facilitan la colonización nasofaríngea.
    \end{itemize}
    \item \textbf{Poblaciones vulnerables}: Lactantes, adultos mayores (>65 años) e inmunocomprometidos.
\end{itemize}

\subsection{Importancia de la vacunación}
La amplia distribución de serotipos y la emergencia de cepas resistentes a antibióticos (ej. penicilina, macrólidos) subrayan la necesidad de estrategias preventivas. Las vacunas conjugadas han reducido la incidencia de enfermedad neumocócica al inducir inmunidad específica contra los serotipos más prevalentes.

\section{Inmunología básica}

El sistema inmunitario tiene dos componentes principales que responden a patógenos como \textit{Streptococcus pneumoniae}:

\begin{itemize}
    \item \textbf{Sistema innato}: Actúa como primera línea de defensa con componentes celulares (neutrófilos, macrófagos y células dendríticas) y humorales (sistema del complemento).
    
    \item \textbf{Sistema adquirido}: Proporciona especificidad y memoria inmunológica, con linfocitos T (CD4+, CD8+) y B como actores principales.
\end{itemize}

\subsection{Respuesta inmune a \textit{Streptococcus pneumoniae}: rol de linfocitos B, memoria inmunológica y producción de anticuerpos}

Los linfocitos B son esenciales en la respuesta a vacunas antineumocócicas:
\begin{itemize}
    \item Producen anticuerpos específicos tras la activación.
    \item Experimentan maduración de afinidad y cambio de isotipo (ej. IgM → IgG).
    \item Generan células de memoria para respuestas futuras más rápidas y eficientes.
\end{itemize}

\subsection{Características únicas en lactantes}
\begin{itemize}
    \item \textbf{Efecto de anticuerpos maternos}: 
    \begin{itemize}
        \item Transferencia placentaria de IgG (especialmente IgG1) mediante el receptor FcRn.
        \item Protección adicional mediante IgA secretora en la lactancia.
        \item Posible \textit{efecto blunting} (interferencia de anticuerpos maternos con la respuesta infantil), aunque su relevancia clínica es limitada.
    \end{itemize}
    
    \item \textbf{Desarrollo del sistema inmune}: 
    % COMENTARIO: El documento no proporciona detalles específicos sobre el desarrollo en los primeros 24 meses
    El sistema inmunitario fetal es funcional pero inmaduro, dependiendo inicialmente de la protección materna.
\end{itemize}

\subsection{Definiciones clave} 
\begin{itemize}
    \item \textbf{Inmunogenicidad vs. efectividad clínica}:
    \begin{itemize}
        \item Inmunogenicidad: Capacidad de una vacuna para inducir una respuesta inmunitaria (ej. producción de anticuerpos).
        \item Efectividad clínica: Protección real contra la enfermedad en condiciones del mundo real.
    \end{itemize}
    
    \item \textbf{Conjugación proteica}:
    % COMENTARIO: El documento no especifica qué proteínas carrier se usan en vacunas antineumocócicas
    Las vacunas conjugadas unen polisacáridos a proteínas carrier para mejorar la respuesta inmunitaria, especialmente en niños.
\end{itemize}

\section{Vacunas antineumocócicas conjugadas}

\subsection{Mecanismo de acción}
Las vacunas antineumocócicas conjugadas utilizan polisacáridos capsulares conjugados a proteínas para:
\begin{itemize}
    \item Inducir una respuesta inmunitaria más robusta y duradera.
    \item Generar memoria inmunológica mediante la activación de linfocitos T y B.
\end{itemize}

% \subsection{Serotipos cubiertos y correlatos de protección}
% \begin{itemize}
%     \item \textbf{PCV11}: Correlatos de protección varían por serotipo (ej.$ ≥ 0.35$ mcg/ml para serotipos 1, 14, etc.; $≥ 0.10$ mcg/ml para 6B).
%     \item \textbf{PCV20}: Mismos correlatos que PCV11 para serotipos compartidos.
%     \item \textbf{Nota}: Estos valores son criterios utilizados en ensayos clínicos (ej. por la FDA), pero la OMS no ha establecido un correlato de protección universal.
% \end{itemize}

% COMENTARIO: El documento no menciona específicamente PCV10 vs. PCV13 ni datos de epidemiología local

\section{Modelado computacional aplicado}

\subsection{Tipos de modelos}
\begin{itemize}
    \item \textbf{Modelos estocásticos}: 
    \begin{itemize}
        \item Útiles para simular interacciones célula-célula (ej. autómatas celulares).
        \item Capturan la variabilidad inherente a sistemas biológicos.
    \end{itemize}
    
    \item \textbf{Modelos deterministas}:
    \begin{itemize}
        \item Ecuaciones diferenciales ordinarias (ODE) para dinámica de poblaciones de linfocitos.
        \item Ideales para sistemas con grandes números de células.
    \end{itemize}
\end{itemize}

