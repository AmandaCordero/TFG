%===================================================================================
% Chapter: Marco Teorico
%===================================================================================
\chapter{Preliminares}\label{chapter:marcoteorico}
%\addcontentsline{toc}{chapter}{Marco Teórico}
%===================================================================================


\textit{Streptococcus pneumoniae} (neumococo) es una bacteria Gram-positiva, encapsulada y con forma lanceolada, que coloniza las vías respiratorias superiores en humanos. Es un patógeno de relevancia clínica global, asociado tanto a enfermedades invasivas (como meningitis, neumonía bacteriémica y sepsis) como a infecciones no invasivas (otitis media aguda y sinusitis).

\section{Características clave}
\begin{itemize}
    \item \textbf{Variabilidad serotípica}: Existen más de 90 serotipos diferenciados por la composición de su cápsula polisacárida, aunque aproximadamente 20 son responsables de la mayoría de los casos de enfermedad invasiva.
    \item \textbf{Mecanismos de patogenicidad}:
    \begin{itemize}
        \item Evasión inmunológica mediada por la cápsula polisacárida, que dificulta la fagocitosis.
        \item Producción de toxinas como la neumolisina, que daña las células del hospedero.
        \item Presencia de adhesinas que facilitan la colonización de la nasofaringe.
    \end{itemize}
    \item \textbf{Poblaciones vulnerables}: Lactantes, adultos mayores (mayores de 65 años) e individuos inmunocomprometidos presentan mayor riesgo de infección grave.
\end{itemize}

\subsection{Importancia de la vacunación}
La diversidad de serotipos y la emergencia de cepas resistentes a antibióticos subrayan la importancia de estrategias preventivas como la vacunación. Las vacunas conjugadas han demostrado reducir significativamente la incidencia de enfermedad neumocócica al inducir inmunidad específica frente a los serotipos más prevalentes \cite{Snedecor2020a,Snedecor2020b}.

\section{Inmunología básica}

El sistema inmunitario humano responde a \textit{S. pneumoniae} mediante dos grandes componentes:

\begin{itemize}
    \item \textbf{Inmunidad innata}: Primera línea de defensa, compuesta por células como neutrófilos, macrófagos y células dendríticas, así como factores humorales como el sistema del complemento.
    \item \textbf{Inmunidad adaptativa}: Proporciona especificidad y memoria inmunológica, mediada por linfocitos T (CD4$^+$, CD8$^+$) y B.
\end{itemize}

\subsection{Respuesta inmune a \textit{S. pneumoniae}: linfocitos B, memoria inmunológica y anticuerpos}

Los linfocitos B son fundamentales en la defensa frente a \textit{S. pneumoniae}, especialmente tras la vacunación:
\begin{itemize}
    \item Producen anticuerpos específicos que neutralizan al patógeno.
    \item Experimentan maduración de afinidad y cambio de isotipo (por ejemplo, de IgM a IgG).
    \item Generan células de memoria que permiten una respuesta más rápida y eficaz ante exposiciones futuras.
\end{itemize}

\subsection{Características inmunológicas en lactantes}
\begin{itemize}
    \item \textbf{Anticuerpos maternos}: 
    \begin{itemize}
        \item Transferencia placentaria de IgG (especialmente IgG1) mediante el receptor FcRn.
        \item Protección adicional a través de IgA secretora en la leche materna.
        \item Posible \textit{blunting} (interferencia de anticuerpos maternos con la respuesta vacunal), aunque su impacto clínico es limitado.
    \end{itemize}
    \item \textbf{Desarrollo inmune}: El sistema inmunitario neonatal es funcional pero inmaduro, dependiendo inicialmente de la inmunidad materna.
\end{itemize}

\subsection{Definiciones clave}
\begin{itemize}
    \item \textbf{Inmunogenicidad}: Capacidad de una vacuna para inducir una respuesta inmunitaria medible (por ejemplo, producción de anticuerpos).
    \item \textbf{Efectividad clínica}: Protección real conferida contra la enfermedad en condiciones del mundo real.
    \item \textbf{Conjugación proteica}: Las vacunas conjugadas unen polisacáridos capsulares a proteínas transportadoras para potenciar la respuesta inmunitaria, especialmente en niños pequeños.
\end{itemize}

\section{Vacunas antineumocócicas conjugadas}


Las vacunas antineumocócicas conjugadas (PCV) emplean polisacáridos capsulares conjugados a proteínas para:
\begin{itemize}
    \item Inducir una respuesta inmunitaria robusta y duradera.
    \item Generar memoria inmunológica mediante la activación de linfocitos T y B.
\end{itemize}

\subsection{Cobertura y correlatos de protección}
Las formulaciones actuales cubren los serotipos más prevalentes y virulentos. Los correlatos de protección pueden variar según el serotipo y la vacuna específica, y son utilizados como criterios en ensayos clínicos y en la evaluación de programas de vacunación \cite{Snedecor2020a}.

% \section{Modelado computacional aplicado}

% El modelado computacional es esencial para comprender y predecir la respuesta inmune a las vacunas neumocócicas, tanto a nivel individual como poblacional. Los enfoques más relevantes identificados en la literatura reciente incluyen:

% \subsection{Modelos dinámicos de transmisión}
% Los modelos dinámicos de transmisión, basados en ecuaciones diferenciales, permiten simular la propagación de la enfermedad y el impacto de la vacunación en diferentes grupos poblacionales. Estos modelos suelen incorporar:
% \begin{itemize}
%     \item \textbf{Estratificación por edad}: (<2, 2-4, 5-17, 18-49, 50-64, $\geq$65 años).
%     \item \textbf{Categorías de serotipos}: PCV7-tipo, PCV6-tipo y serotipos no incluidos en la vacuna.
%     \item \textbf{Parámetros epidemiológicos}: Adquisición de portadores, transmisión, desarrollo de enfermedad invasiva.
% \end{itemize}
% La validación de estos modelos se realiza mediante la comparación con datos epidemiológicos observados a lo largo de varios años, mostrando una alta concordancia entre los casos modelados y los reales, lo que respalda su utilidad para predecir el impacto de la vacunación y apoyar decisiones de salud pública \cite{Snedecor2020a,Snedecor2020b}.

% \subsection{Enfoques de inmunología de sistemas}
% A nivel individual, los enfoques de inmunología de sistemas emplean herramientas computacionales para analizar firmas inmunes tempranas y su relación con la respuesta vacunal. Por ejemplo, el análisis de perfiles transcripcionales y la dinámica de plasmablastos tras la vacunación permiten identificar biomarcadores predictivos de eficacia inmune, lo que abre la puerta a estrategias de vacunación personalizadas \cite{Obermoser2013}.

% \subsection{Tipos de modelos matemáticos}
% \begin{itemize}
%     \item \textbf{Modelos deterministas}: Utilizan ecuaciones diferenciales ordinarias (ODE) para describir la dinámica de poblaciones celulares y la progresión de la enfermedad.
%     \item \textbf{Modelos estocásticos}: Simulan interacciones a nivel de célula individual (por ejemplo, autómatas celulares), capturando la variabilidad inherente a los sistemas biológicos.
% \end{itemize}

% \subsection{Aplicaciones y predicción a largo plazo}
% Los modelos computacionales han demostrado ser capaces de predecir con precisión la incidencia de enfermedad neumocócica invasiva (IPD) durante periodos prolongados (hasta 17 años), así como de evaluar el impacto de cambios en los programas de vacunación (por ejemplo, la transición de PCV7 a PCV13 o la inclusión de adultos en la estrategia vacunal) \cite{Snedecor2020a,Snedecor2020b}.

% \section{Conclusiones de los enfoques computacionales}

% La integración de modelos dinámicos de transmisión y herramientas de inmunología de sistemas proporciona una visión integral del impacto de las vacunas neumocócicas, permitiendo tanto la optimización de políticas de salud pública como el avance hacia una medicina más personalizada.


