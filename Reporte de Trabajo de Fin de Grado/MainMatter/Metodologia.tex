%===================================================================================
% Chapter: Metodologia
%===================================================================================
\chapter{Metodología}\label{chapter:metodologia}
%\addcontentsline{toc}{chapter}{Marco Teórico}
%===================================================================================

En este capítulo se describe la metodología y la estructura empleada en la implementación de un simulador computacional del sistema inmune de un lactante para modelar la respuesta a vacunas conjugadas neumocócicas.

\section{Parámetros}
\begin{table}[H]
\centering
\caption{Parámetros utilizados en la simulación del sistema inmune.}
\label{tab:parametros_simulacion}
\begin{tabular}{|l|p{2cm}|p{5cm}|}
\hline
\textbf{Parámetro} & \textbf{Tipo} & \textbf{Descripción} \\ \hline
duration\_days & Entero & Duración total de la simulación en días. \\ \hline
vaccination\_schedule & Lista de enteros & Días en los que se aplican las vacunas (dosis 1, 2 y refuerzo). \\ \hline
lf\_decay & Flotante & Tasa de decaimiento natural de linfocitos no activos. \\ \hline
decay\_factor & Flotante & Factor de decaimiento de anticuerpos. \\ \hline
plasma\_production\_factor & Diccionario (str: float) & Tasa de producción de anticuerpos a partir de células plasmáticas por cada serotipo. \\ \hline
memory\_production\_factor & Diccionario (str: float) & Tasa de producción de anticuerpos a partir de células de memoria activadas por cada serotipo. \\ \hline
\end{tabular}
\end{table}


\begin{table}[H]
\centering
\caption{Parámetros utilizados en la simulación del sistema inmune.}
\label{tab:parametros_simulacion2}
\begin{tabular}{|l|p{2cm}|p{5cm}|}
\hline
\textbf{Parámetro} & \textbf{Tipo} & \textbf{Descripción} \\ \hline

affinity\_threshold\_memory & Flotante & Umbral mínimo de afinidad para que una célula B se convierta en célula de memoria. \\ \hline
affinity\_threshold\_plasma & Flotante & Umbral mínimo de afinidad para que una célula B se diferencie en célula plasmática. \\ \hline
mutation\_p & Flotante & Probabilidad de mutación de una célula B. \\ \hline
mutation\_rate & Flotante & probabilidad de mutar cada componente del vector de receptores. \\ \hline
mutation\_strength & Flotante & Intensidad del cambio en la afinidad tras una mutación celular. \\ \hline
temperature & Flotante & regula la presión selectiva. \\ \hline
THRESHOLD & Flotante & Umbral mínimo de afinidad para que una célula B entre a un centro germinal. \\ \hline
\end{tabular}
\end{table}


\section{Estructura}
El simulador se estructura en dos capas principales:
\begin{itemize}
    \item Modelado de componentes biológicos:
        \begin{itemize}
            \item Antígenos
            \item Células B
        \end{itemize}
    \item Entornos de Interacción:
        \begin{itemize}
            \item Centros germinales
            \item Sistema inmune
        \end{itemize}
\end{itemize}
\section{Modelado de componentes biológicos}
\begin{enumerate}
    
    \item \texttt{Antígeno}: Cualquier sustancia que haga que el cuerpo produzca una respuesta inmunitaria contra ella. Conformado por epítopo\footnote{Parte de una molécula que será reconocida por un anticuerpo y a la cual se unirá.} y una proteína portadora. En el estudio este se modela como un objeto con propiedades: \texttt{id} (entero que representa el identificador del antígeno), \texttt{serotype} (\textit{string} que representa código del serotipo), \texttt{carrier\_protein} (\textit{string} que representa la proteína portadora) y \texttt{epitope\_vector} (vector de \textit{float} que representa el epítopo).
    \item \texttt{Célula B}: Cualquier linfocito encargado de reconocer y responder a un antígeno específico mediante la producción de anticuerpos. Conformada por su receptor BCR  (receptor de célula B), que reconoce y se une a un epítopo  particular. En el estudio, esta se modela como un objeto con propiedades: \texttt{id} (entero que representa el identificador de la célula), \texttt{bcr\_vector} (vector de \textit{float} que representa el  BCR), \texttt{associated\_serotype} (\textit{string} que representa código del serotipo asociado al antígeno reconocido) y \texttt{affinity} (\textit{float} que represneta la afinidad de unión al epítopo).
\end{enumerate}

    


\section{Centros Germinales}

Los centros germinales constituyen conglomerados de células B que reconocen el mismo antígeno en los cuales ocurren diferentes procesos inmunológicos.

    \subsection*{Cálculo de afinidad:}

    La afinidad entre una célula B y un antígeno es una medida fundamental que cuantifica la fuerza de interacción entre el BCR y el epítopo del antígeno. Este cálculo se basa en la similitud estructural entre sus vectores característicos.

    Definimos la afinidad $A$ mediante la transformación exponencial de la distancia euclidiana entre los vectores:

    \begin{equation}
    A = e^{-d(\vec{r}_{B}, \vec{e}_{A})} = \exp\left({-||\vec{r}_{B} - \vec{e}_{A}||}\right)
    \end{equation}

    donde:
    \begin{itemize}
        \item $\vec{r}_B \in \mathbb{R}^n$ es el vector característico del receptor de la célula B, que codifica las propiedades moleculares de su sitio de unión.
        
        \item $\vec{e}_A \in \mathbb{R}^n$ es el vector característico del epítopo antigénico, representando sus características estructurales clave para el reconocimiento inmunológico.
        
        \item $d(\vec{r}_{B}, \vec{e}_{A}) = ||\vec{r}_{B} - \vec{e}_{A}||$ es la distancia euclidiana en $\mathbb{R}^n$, calculada como:
        \begin{equation}
        d = \sqrt{\sum_{i=1}^{n} (r_{B,i} - e_{A,i})^2}
        \end{equation}
        
        \item $A \in (0,1]$ es la medida de afinidad resultante.
    %     , con propiedades biológicamente significativas:
    %     \begin{itemize}
    %         \item \textbf{Valor 1}: Máxima afinidad (vectores idénticos, distancia = 0)
    %         \item \textbf{Valores cercanos a 0}: Baja afinidad (vectores disímiles)
    %         \item \textbf{Función monótonamente decreciente}: Mayor distancia $\Rightarrow$ menor afinidad
    %     \end{itemize}
    \end{itemize}

    % \textbf{Interpretación biológica:} Esta formulación captura dos principios esenciales de las interacciones inmunológicas:
    % \begin{enumerate}
    %     \item La \textbf{complementariedad estructural} entre el BCR y el epítopo determina la fuerza de unión
    %     \item La relación no es lineal: pequeñas diferencias estructurales en regiones críticas pueden causar grandes disminuciones en la afinidad (propiedad emergente de la función exponencial)
    % \end{enumerate}

    % \textbf{Ventajas computacionales:}
    % \begin{itemize}
    %     \item Normalización intrínseca ($0 < A \leq 1$)
    %     \item Diferenciabilidad para algoritmos de optimización
    %     \item Penalización exponencial de disparidades estructurales
    %     \item Eficiencia de cálculo con operaciones vectorizadas
    % \end{itemize}
    \subsection*{Selección de células B mediante distribución de Boltzmann:}

La selección de las células B que sobreviven en cada generación se realiza mediante un algoritmo estocástico basado en la distribución de Boltzmann, que asigna probabilidades de supervivencia proporcionales a la afinidad de cada célula mientras mantiene diversidad poblacional. Este proceso consta de tres etapas:

\begin{enumerate}
    \item \textbf{Cálculo de probabilidades:} Para cada célula B con afinidad $A_i$, se calcula una probabilidad de supervivencia:
    \[
    P_i = \frac{e^{A_i / T}}{\sum_{j=1}^{N} e^{A_j / T}}
    \]
    donde:
    \begin{itemize}
        \item $T$ es el parámetro de temperatura que regula la presión selectiva
        \item $N$ es el tamaño total de la población
        \item $A_i \in [0,1]$ es la afinidad de la célula $i$
    \end{itemize}
    
    \item \textbf{Normalización numérica:} Para evitar inestabilidades computacionales, se aplica una transformación estable:
    \[
    A_i^{\text{norm}} = A_i - \max_{j}(A_j)
    \]
    garantizando valores negativos o cero sin alterar las proporciones relativas.
    
    \item \textbf{Muestreo estocástico:} Se seleccionan $K$ células sin reemplazo según:
    \[
    \text{Seleccionados} \sim \text{Multinomial}(K; P_1, P_2, \dots, P_N)
    \]
    donde $K = \min(\text{max\_survivors}, N)$ modela restricciones de recursos.
\end{enumerate}

\noindent \textbf{Interpretación de parámetros:}
\begin{itemize}
    \item \textbf{Temperatura (T):} Regula el balance diversidad-optimización:
    \begin{itemize}
        \item $T \to 0^+$: Selección elitista (solo células con máxima afinidad)
        \item $T \to \infty$: Selección aleatoria uniforme (máxima diversidad)
        \item $T \approx 0.5$: Balance biológico típico en simulaciones
    \end{itemize}
    
    \item \textbf{max\_survivors:} Modela restricciones biológicas:
    \begin{itemize}
        \item Espacio en nichos linfoides
        \item Disponibilidad de factores de crecimiento
        \item Capacidad de presentación antigénica
    \end{itemize}
\end{itemize}

% \noindent \textbf{Base biológica:} Este modelo captura mecanismos esenciales de selección clonal:
% \begin{itemize}
%     \item \textbf{Supervivencia diferencial:} Células con mayor afinidad tienen mayor probabilidad de persistir
%     \item \textbf{Preservación de diversidad:} Permite la supervivencia de células con afinidad subóptima pero potencial evolutivo
%     \item \textbf{Competición por recursos:} El muestreo sin reemplazo simula limitaciones del microambiente
% \end{itemize}

% \noindent \textbf{Ventajas computacionales:}
% \begin{itemize}
%     \item Diferenciabilidad para métodos basados en gradiente
%     \item Complejidad $O(N)$ eficiente para grandes poblaciones
%     \item Control preciso de presión selectiva mediante $T$
%     \item Implementación estable con normalización exponencial
% \end{itemize}
    \subsection*{Diferenciación:}
    
    Según su afinidad, las células B pueden diferenciarse en células plasmáticas o de memoria \cite{murphy2017janeway}.

    En la simulación se establecen umbrales en la afinidad para esta clasificación.
    \subsection*{Hipermutación somática de células B:}

El proceso de mutación simula la hipermutación somática que ocurre en los centros germinales durante la respuesta inmune adaptativa. Cada componente del vector receptor $\vec{r}_B$ de una célula B tiene una probabilidad $p_m$ de sufrir una modificación estocástica, modelada como:

\begin{equation}
r_i' = \text{clip}\left(r_i + \delta_i, 0, 1\right) \quad \text{donde} \quad \delta_i \sim 
\begin{cases} 
\mathcal{N}(0, \sigma^2) & \text{con probabilidad } p_m \\
0 & \text{en caso contrario}
\end{cases}
\end{equation}

\noindent \textbf{Parámetros clave:}
\begin{itemize}
    \item \textit{Tasa de mutación ($p_m$)}: Probabilidad de que un componente individual mute (valores típicos: 0.05-0.15). Biológicamente, refleja la actividad de la enzima AID (citidina desaminasa) en linfocitos B.
    
    \item \textit{Intensidad de mutación ($\sigma$)}: Desviación estándar de la distribución gaussiana que determina la magnitud de las alteraciones (valores típicos: 0.01-0.1). Modela el impacto estructural de las mutaciones puntuales.
\end{itemize}

\noindent \textbf{Procesamiento post-mutación:}
\begin{itemize}
    \item \textit{Normalización de valores:} Los componentes se limitan al intervalo $[0,1]$ mediante:
    \[
    \text{clip}(x, a, b) = \max(a, \min(x, b))
    \]
    garantizando vectores biológicamente plausibles.
    
    \item \textit{Reseteo de afinidad:} La afinidad de la célula mutada se inicializa a cero, requiriendo recálculo contra antígenos.
    
    \item \textit{Preservación de linaje:} Se mantienen el identificador único y serotipo para rastreo evolutivo.
\end{itemize}

\noindent \textbf{Base biológica:} Este modelo captura tres propiedades esenciales de la hipermutación somática:
\begin{enumerate}
    \item \textit{Localidad:} Mutaciones puntuales que modifican ligeramente el paratopo
    
    \item \textit{Independencia:} Cada residuo muta de forma autónoma
    
    \item \textit{No direccionalidad:} Los cambios son estocásticos, no guiados por el antígeno
\end{enumerate}

 
% \noindent \textbf{Ejemplo numérico:} Para un receptor $[0.8, 0.5]$ con $p_m=0.1$, $\sigma=0.05$:
% \begin{itemize}
%     \item Componente 1: No muta (prob $>0.1$) $\rightarrow 0.8$
%     \item Componente 2: Mutación con $\delta = +0.03$ $\rightarrow 0.53$
%     \item Receptor mutado: $[0.8, 0.53]$
% \end{itemize}


Las células B ingresan al centro germinal tras calcular su afinidad inicial con el antígeno. Dentro del mismo, el proceso de selección identifica las células candidatas para abandonarlo, las cuales se diferencian en células plasmáticas o de memoria antes de reintegrarse al sistema inmunológico. Las células no seleccionadas enfrentan dos destinos posibles: muerte por apoptosis (pérdida energética), o supervivencia con división y mutación. Para este último grupo, se aplica hipermutación somática a sus receptores y se recalcula su afinidad antigénica, reiniciando el ciclo, como se muestra en la figura \ref{fig:cg}. Este proceso iterativo continúa hasta la desaparición de todas las células del centro germinal. 

\begin{figure}[H] % h = here (aquí)
    \centering
    \includegraphics[width=1\textwidth]{Graphics/gc.png}
    \caption{Ciclo que ocurre dentro del Centro Germinal}
    \label{fig:cg}
\end{figure}

\section{Sistema Inmune}

El sistema inmune consta de un grupo de células B \textit{naive} que se dividen y mutan o mueren. Ante la presencia de antígeno estas células pueden pasar a formar parte de centros germinales. 

\subsection*{Proceso de llenado de los centros germinales}
Las células B actúan como recursos que pueden ser reclutados por los antígenos (según su afinidad por el mismo) para formar parte de un centro germinal. En el modelo, se crea un hilo (\textit{thread}) por cada antígeno para simular de forma concurrente el proceso de selección de células B en los centros germinales. Este enfoque permite que todos los antígenos tengan igual prioridad al acceder a dichos recursos. Sin embargo, surge el problema de que dos antígenos podrían intentar reclutar la misma célula B simultáneamente. Para evitar esta condición de carrera, se implementa un mecanismo de exclusión mutua (\textit{mutex}) que garantiza que solo un antígeno pueda analizar una célula B en un momento dado.

Al finalizar cada ciclo, los centros germinales liberan al sistema inmunológico células B diferenciadas en células plasmáticas y células de memoria. La concentración de anticuerpos generada por estas sirve como indicador cuantitativo de la respuesta inmunológica del individuo. Este proceso se muestra en la figura \ref{fig:is}.

\begin{figure}[H] % h = here (aquí)
    \centering
    \includegraphics[width=1\textwidth]{Graphics/is.png}
    \caption{Células B dentro del sistema inmune}
    \label{fig:is}
\end{figure}

\section{Salida del simulador}
La salida del simulador consiste en una serie de tiempo que registra diariamente el nivel de anticuerpos específicos para cada serotipo presentes en el organismo durante toda la simulación. Este nivel se interpreta como una medida cuantitativa del resultado que se obtendría al realizar una prueba ELISA.
Cada día de la simulación, se almacena el valor del nivel de anticuerpos específicos para el serotipo, calculado como la suma ponderada de las contribuciones individuales de las células productoras de anticuerpos (células de plasma y memoria).
     



% aaaaaaaaaaaaaaaaaaaaaaaaaaaaaaaaaaaaaaaaaaaaaaaaaaaaaaaaaaaaaaaaaaaaaaaaaaaaaaaaaaa
% aaaaaaaaaaaaaaaaaaaaaaaaaaaaaaaaaaaaaaaaaaaaaaaaaaaaaaaaaaaaaaaaaaaaaaaaaaaaaaaaaaa
% aaaaaaaaaaaaaaaaaaaaaaaaaaaaaaaaaaaaaaaaaaaaaaaaaaaaaaaaaaaaaaaaaaaaaaaaaaaaaaaaaaa
% aaaaaaaaaaaaaaaaaaaaaaaaaaaaaaaaaaaaaaaaaaaaaaaaaaaaaaaaaaaaaaaaaaaaaaaaaaaaaaaaaaa
% aaaaaaaaaaaaaaaaaaaaaaaaaaaaaaaaaaaaaaaaaaaaaaaaaaaaaaaaaaaaaaaaaaaaaaaaaaaaaaaaaaa


