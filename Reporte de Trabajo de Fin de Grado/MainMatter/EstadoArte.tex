 
%===================================================================================
% Chapter: Estado del Arte
%===================================================================================
\chapter{Estado del Arte}\label{chapter:estadoarte}
%\addcontentsline{toc}{chapter}{Estado del Arte}
%===================================================================================

La modelización computacional ha emergido como una herramienta fundamental para comprender y predecir las complejas interacciones entre las vacunas neumocócicas conjugadas (PCV) y el sistema inmunológico humano. Los documentos analizados revelan metodologías complementarias que abordan este fenómeno desde perspectivas poblacionales e individuales, proporcionando un marco integral para el desarrollo de modelos predictivos eficaces.

\section{Enfoques Computacionales Predominantes en la Modelización Inmunológica}

La literatura científica actual identifica dos metodologías computacionales principales para modelar la respuesta inmune a las vacunas neumocócicas conjugadas, cada una con fortalezas y aplicaciones distintivas que contribuyen a nuestra comprensión global de este complejo fenómeno biológico.

\subsection{Modelos de Transmisión Dinámica}

Los modelos de transmisión dinámica basados en ecuaciones diferenciales constituyen uno de los enfoques fundamentales para analizar el impacto de las vacunas neumocócicas a nivel poblacional. Estos modelos matemáticos utilizan sistemas de ecuaciones diferenciales para simular la propagación de la enfermedad neumocócica invasiva (IPD) a través de poblaciones estratificadas. Los estudios de Snedecor \cite{Snedecor2020a,Snedecor2020b} demuestran que estos modelos incorporan múltiples componentes críticos para garantizar su precisión predictiva.

La estratificación por edad es un elemento esencial de estos modelos, dividiendo la población en categorías específicas ($<$2, 2-4, 5-17, 18-49, 50-64, $\geq$65 años) que reflejan las diferencias en susceptibilidad y transmisión entre diversos grupos etarios. Esta segmentación permite capturar tanto los efectos directos como indirectos de la vacunación en diferentes cohortes poblacionales.

Adicionalmente, estos modelos categorizan los serotipos neumocócicos según su inclusión en diversas formulaciones vacunales (PCV7-tipo, PCV6-tipo, y no-PCV-tipo), facilitando la evaluación del impacto diferencial de las vacunas en la circulación de serotipos específicos. La incorporación de parámetros de adquisición de portadores y desarrollo de enfermedad invasiva enriquece aún más la capacidad predictiva de estos modelos.

\subsection{Enfoques de Inmunología de Sistemas}

En contraste con los modelos poblacionales, los enfoques de inmunología de sistemas se centran en las respuestas inmunológicas individuales a la vacunación neumocócica. El estudio de Obermoser et al. \cite{Obermoser2013} ejemplifica esta metodología, aplicando soluciones de software para vincular firmas inmunes tempranas con respuestas vacunales posteriores.

Estos enfoques analizan cambios transcripcionales post-vacunación, estableciendo correlaciones entre estas medidas y los títulos de anticuerpos desarrollados al día 28. La medición de respuestas de plasmablastos al día 7 proporciona información adicional sobre la dinámica de la respuesta inmune adaptativa, creando un marco temporal para la predicción de eficacia vacunal a nivel individual.

Las soluciones computacionales empleadas en inmunología de sistemas permiten el procesamiento de conjuntos de datos inmunológicos complejos, facilitando la identificación de biomarcadores predictivos de respuesta vacunal. Esta capacidad tiene implicaciones significativas para la personalización de estrategias de vacunación según perfiles inmunológicos individuales.

\section{Integración y Validación de Modelos Computacionales}

La robustez de los modelos computacionales depende fundamentalmente de procesos de validación rigurosos que confirmen su capacidad predictiva frente a datos observados en contextos reales.

\subsection{Validación con Datos Epidemiológicos}

Los modelos de transmisión dinámica desarrollados por Snedecor \cite{Snedecor2020a,Snedecor2020b} fueron validados mediante comparación con datos epidemiológicos observados durante un período de 17 años (2000-2016). Esta extensa validación temporal demuestra la capacidad de estos modelos para capturar efectos a largo plazo de las estrategias de vacunación neumocócica.

Los resultados reportados muestran una notable concordancia entre los casos modelados y observados: para niños menores de 2 años, el modelo predijo 18.5 casos frente a 13.5 casos observados; para adultos de 65 años o más, predijo 23.6 casos frente a 24 casos observados. Esta precisión refuerza la validez de los modelos para informar políticas de salud pública relacionadas con la vacunación neumocócica.

\subsection{Correlación Inmunológica y Validación Clínica}

En el ámbito de la inmunología de sistemas, la validación se centra en correlaciones entre biomarcadores tempranos y respuestas inmunes posteriores. El estudio analizado reporta correlaciones entre medidas transcripcionales tempranas y títulos de anticuerpos al día 28, así como la medición de respuestas adaptativas mediante la dinámica de plasmablastos al día 7 \cite{Obermoser2013}.

Esta validación proporciona un fundamento para el desarrollo de modelos predictivos de respuesta individual a la vacunación, complementando la perspectiva poblacional ofrecida por los modelos de transmisión dinámica.

\section{Aplicaciones de los Modelos en Estrategias de Vacunación}

Los modelos computacionales validados ofrecen valiosas aplicaciones para la planificación e implementación de estrategias de vacunación neumocócica a diferentes niveles.

\subsection{Predicción del Impacto Poblacional}

Los modelos de transmisión dinámica han demostrado capacidad para prever el impacto de programas de vacunación a largo plazo, incluyendo efectos directos en poblaciones vacunadas y efectos indirectos mediante inmunidad de grupo. La capacidad de estos modelos para predecir con precisión los cambios en la incidencia de IPD durante 17 años subraya su utilidad para la planificación sanitaria estratégica.

El segundo documento analizado sugiere que los modelos computacionales estructurados por edad integran datos de vigilancia y estimaciones epidemiológicas para predecir específicamente las respuestas a la vacuna PCV en recién nacidos, demostrando reducciones en la transmisión de enfermedad y susceptibilidad. Esta aplicación específica ilustra la versatilidad de estos modelos para evaluar intervenciones en poblaciones vulnerables.

\subsection{Optimización de Programas Vacunales}

La capacidad predictiva de estos modelos facilita la optimización de programas de vacunación existentes. Los estudios analizados indican que los modelos han predicho exitosamente el impacto de transiciones entre formulaciones vacunales (PCV7 a PCV13) y los efectos de ampliar la vacunación a poblaciones adultas.

Esta capacidad permite a los responsables de políticas sanitarias evaluar múltiples escenarios de implementación antes de realizar cambios programáticos, maximizando el impacto de recursos limitados y minimizando consecuencias no intencionadas.

\section{Respuestas Inmunológicas Específicas a Vacunas Neumocócicas}

Los documentos analizados ofrecen información valiosa sobre la modelización de respuestas inmunológicas específicas desencadenadas por vacunas neumocócicas, tanto a nivel innato como adaptativo.

\subsection{Modelización de Respuestas Inmunes Innatas}

El estudio de Obermoser et al. \cite{Obermoser2013} examinó respuestas inmunes innatas a vacunas neumocócicas dentro de las horas posteriores a la vacunación mediante análisis de firmas transcripcionales. Aunque la información disponible no detalla completamente la integración de estos datos en modelos computacionales, este enfoque proporciona \textit{insights} valiosos sobre los eventos inmunológicos tempranos que pueden influir en la eficacia vacunal.

La modelización de estas respuestas innatas tempranas representa un área prometedora para comprender determinantes precoces de la respuesta vacunal y potencialmente identificar individuos con mayor probabilidad de desarrollar protección eficaz.

\subsection{Caracterización de Respuestas Inmunes Adaptativas}

Las respuestas inmunes adaptativas fueron evaluadas al día 7 post-vacunación mediante medición de respuestas de plasmablastos, estableciendo correlaciones con títulos de anticuerpos posteriores. Esta caracterización temporal de la respuesta adaptativa proporciona un marco para predecir la eficacia vacunal antes de que se establezcan respuestas de anticuerpos completas.

La modelización de estas respuestas adaptativas complementa los modelos poblacionales al proporcionar mecanismos inmunológicos subyacentes que explican las tasas de protección observadas a nivel epidemiológico.

\section{Avances en la Modelización de Poblaciones Específicas}

Los documentos analizados revelan avances significativos en la modelización computacional dirigida a poblaciones específicas, particularmente recién nacidos y niños pequeños, que representan grupos prioritarios para la vacunación neumocócica.

\subsection{Modelización en Población Neonatal e Infantil}

El segundo documento indica que modelos computacionales estructurados por edad integran datos de vigilancia y estimaciones epidemiológicas para predecir respuestas a PCV en recién nacidos. Estos modelos demuestran reducciones en transmisión y susceptibilidad a enfermedad, proporcionando evidencia computacional para apoyar políticas de vacunación temprana.

Esta modelización específica para recién nacidos es particularmente valiosa considerando las características únicas del sistema inmune neonatal y las complejas interacciones con anticuerpos maternos, factores que pueden influir significativamente en la respuesta vacunal.

\subsection{Comparaciones entre Grupos Etarios}

Los modelos de transmisión dinámica analizados incorporan múltiples grupos etarios, permitiendo evaluaciones comparativas de la efectividad vacunal entre diferentes cohortes poblacionales. Esta estratificación facilita la identificación de grupos con respuestas subóptimas que podrían beneficiarse de estrategias vacunales adaptadas.

La capacidad de estos modelos para predecir con precisión casos de IPD en grupos etarios específicos (niños menores de 2 años y adultos de 65 años o más) demuestra su utilidad para informar políticas vacunales diferenciadas según edad.

% \section{Limitaciones y Direcciones Futuras en la Modelización Computacional}

% A pesar de los avances significativos, el análisis de los documentos revela brechas importantes y oportunidades para el desarrollo futuro de modelos computacionales de respuesta inmune a vacunas neumocócicas.

% \subsection{Integración de Múltiples Escalas Modelísticas}

% Una limitación actual es la separación relativa entre modelos poblacionales (transmisión dinámica) y modelos individuales (inmunología de sistemas). El desarrollo de marcos integrados que vinculen mecánicamente estos niveles representaría un avance significativo, permitiendo predicciones más precisas que incorporen tanto mecanismos inmunológicos detallados como dinámicas poblacionales.

% Esta integración multiescala podría facilitar la identificación de mecanismos inmunológicos subyacentes responsables de tendencias epidemiológicas observadas y mejorar la capacidad para predecir impactos de nuevas formulaciones vacunales.

% \subsection{Expansión a Poblaciones Diversas}

% Aunque los documentos analizados proporcionan información valiosa sobre modelización en recién nacidos y diversos grupos etarios, existe oportunidad para expandir estos modelos a poblaciones con características inmunológicas especiales, como individuos inmunocomprometidos, poblaciones con alta carga de comorbilidades o grupos con exposición previa significativa a patógenos neumocócicos.

% La adaptación de modelos computacionales a estas poblaciones diversas mejoraría su aplicabilidad en contextos globales donde factores demográficos, genéticos y ambientales pueden influir significativamente en la respuesta inmune a vacunas neumocócicas.

\section{Conclusión}

El estado actual de la modelización computacional de respuestas inmunes a vacunas neumocócicas conjugadas revela un campo en evolución con enfoques complementarios que abordan diferentes escalas biológicas del fenómeno vacunal. Los modelos de transmisión dinámica proporcionan predicciones robustas a nivel poblacional, mientras que los enfoques de inmunología de sistemas ofrecen \textit{insights} mecanísticos sobre respuestas individuales.

La validación rigurosa de estos modelos contra datos epidemiológicos observados durante períodos extensos demuestra su fiabilidad para informar políticas de salud pública. Sin embargo, persisten oportunidades significativas para integrar múltiples escalas modelísticas y expandir su aplicabilidad a poblaciones diversas.

% El desarrollo futuro de este campo probablemente se dirigirá hacia marcos integrados que incorporen tanto mecanismos inmunológicos detallados como dinámicas poblacionales, mejorando nuestra capacidad para optimizar estrategias de vacunación neumocócica y, en última instancia, reducir la carga global de enfermedad neumocócica invasiva.




